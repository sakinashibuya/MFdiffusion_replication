\documentclass[10pt,letterpaper]{article}
\usepackage[utf8]{inputenc}
\usepackage{pdflscape}
\usepackage{amsmath}
\usepackage{amsmath}
\usepackage{amsfonts}
\usepackage{fullpage}
\usepackage{amssymb}
\usepackage{threeparttable}
\usepackage[dvipdfmx, hiresbb]{graphicx}
\usepackage[outdir=./]{epstopdf}
\usepackage{natbib}
\usepackage{hyperref}
\hypersetup{colorlinks = true,
                citecolor = {black}}
\usepackage{bbm}
\usepackage{multicol}
\usepackage{longtable}
\usepackage{pgf,pgffor}
\usepackage{tikz}
\setcounter{totalnumber}{8}
\author{Mizuhiro Suzuki}
\title{}
\begin{document}

\maketitle

\tableofcontents

\section{}

\begin{itemize}
  \item Main Matlab file: \texttt{"GMMDiffusion/Main\_models\_1\_3"} for model without $\lambda$ (endorsement effects) and \texttt{"GMMDiffusion/Main\_models\_2\_4"} for model with $\lambda$
\end{itemize}

\section{\texttt{Main\_models\_1\_3.m}}\label{main_models_1_3}

\subsection*{Part 0}
\begin{itemize}
  \item 26: 2 model types: modelType == 1 for $q_N = q_P$ and == 3 for $q_N \ne q_P$
\end{itemize}

\subsection*{Part 2}
\begin{itemize}
  \item 33: Different versions of moments
    (In the paper, they say they use 6 moments, but no version says 6 moments, why?)
\end{itemize}

\subsection*{Part 3}
\begin{itemize}
  \item 49: \texttt{TMonths}: number of months in data of each village
  \item 57: \texttt{T}: number of trimesters calculated from \texttt{TMonths}. $ceil(X)$ rounds each element of $X$ to the nearest integer greater than or equal to that element, and $./$ does element-wise right division.
  \item 59: This line checks whether the number of elements of T is equal to the number of graphs. has same $assert(cond)$ throws an error if cond is false. $numel(A)$ returns the number of elements, n, in array A.
\end{itemize}

\subsection*{Part 4}
\begin{itemize}
  \item 63-69: Parameter grids: although explained in detail in the Supplementary Materials, not shown how they did it. The grid of $q_N$ is slightly different from the Supplementary Materials...
\end{itemize}

\subsection*{Part 5}
\begin{itemize}
  \item 90: Load the network data, which contains adjacency matrices of 43 villages. $X$ is 43 by 1 cell.
  \item 95-: For each village,
    \begin{itemize}
      \item 101: Load the leader data: ID and dummy of being a leader or not (only use the latter)
      \item 106: Load the take-up data: Only dummy (maybe the same order as the one above?)
      \item 107: \texttt{EmpRate} is the proportion of non-leaders who took up MF ($\sim$ means what?)
      \item 110: \texttt{inGiant}??? 
        Maybe the biggest cluster in the village? (Yes, belong to largest component or not.)
      \item 113-114: \texttt{d} = degree of network, \texttt{hermits} = households without any connections with other households (= isolates). $sum(A,2)$ is a column vector containing the sum of each row of matrix A (2 is the dimension).
      \item 117: Load the covariates of the households (not sure which column corresponds to which variable, maybe in readme file or somewhere?) (6 covariates, in the order of number of rooms in a household, number of beds, whether the household has private/government/no electricity, whether the household has own/common/no latrine, number of rooms per capita and number of beds per capita according to Readme file)
      \item 120: Select covariates to use (but in the end they used all the covariates in the data?) (Yes, I think so because they set "1:6" (all covariates) in this line)
      \item 123-130: Restrict the sample to households belonging to the giant networks and households of leaders (number of rows will be reduced). logical(A) converts A into an array of logical values. In the subsequent simulations, the authors seem to focus only on the giant component.
        \begin{itemize}
          \item 123: Leaders
          \item 124: Take-up
          \item 125: Covariates
          \item 127: Take-up by leaders
          \item 128: Covariates of leaders
          \item 129: Outcome = Take-up by leaders
          \item 130: Covars = Covariates of leaders
            \begin{itemize}
              \item \texttt{Outcome} and \texttt{Covars} will be used to estimate $\beta$
            \end{itemize}
        \end{itemize}
      \item 133-137: Second neighbors (\texttt{sec}) = matrix indicating households whose distance between them is 2.
        Here, $X^2$ means $X * X$.
        (See \href{https://en.wikipedia.org/wiki/Adjacency_matrix#Matrix_powers}{here})
        \begin{itemize}
		  \item 133: $>0$ turns the elements of the matrix into binary number (0 or 1)
          \item 134-136: Ruling out households themselves (with two steps, they can always come back to themselves)
          \item 137: Ruling out households with distance 1 (think about a triangle)
        \end{itemize}
    \end{itemize}
\end{itemize}

\subsection*{Part 6}
\begin{itemize}
  \item 142-143: Logistic regression by using leaders to estimate $\beta$ (vector of 6 coefficient estimates)
\end{itemize}

\subsection*{Part 7}
\begin{itemize}
  \item 147-160: Calculate the moments in the case where $q_N = q_P$
    \begin{itemize}
      \item 153-160: Calculate moments at each grid of $q_N (= q_P)$
        \begin{itemize}
          \item 154: \texttt{theta} is the choice of parameters $q_N$ and $q_P$
          \item 155: \texttt{tic} starts stopwatch timer (\texttt{toc} ends timer) to measure elapsed time
          \item 156: Call the function \hyperref[divergence_model]{\texttt{divergence\_model}} which calculates moments
        \end{itemize}
    \end{itemize}
  \item 163-178: Calculate moments in the case where $q_N \ne q_P$
    \begin{itemize}
      \item 169-174: Calculate moments at each grid of $(q_N, q_P)$
        \begin{itemize}
          \item 172: Call the function \hyperref[divergence_model]{\texttt{divergence\_model}} which calculates moments
        \end{itemize}
    \end{itemize}
  \item Note: In the calculations here, a matrix $D$ is obtained.
    This is a length($q_N$ grids)-by-length($q_P$ grids) matrix, and each element is a (\# villages)-by-(\# moments) matrix.
  \item 181-183: Save the computed data in the mat format.
\end{itemize}

\subsection*{Part 8}
\begin{itemize}
  \item 188-194: Select if just obtaining point estimates or standard errors through bootstrap
    \begin{itemize}
      \item bootstrap == 0: point estimates
      \item bootstrap == 1: standard errors through bootstrap (1000 times)
    \end{itemize}
  \item 197-207: Select if two step optimal weights are used or not
    \begin{itemize}
      \item twoStepOptimal == 1: 
        \begin{itemize}
          \item 202: With some guess of $q_N (= 0.09)$ and $q_P (= 0.45)$, obtain moments by the function \texttt{divergence\_model}
          \item 203: Calculate the outer product of the moments, divided by the number of villages
          \item 204: Take the inverse of the above expression, which will be the weight in the second step
          \item \textbf{Note}: I guess usually people 
            (i)solve the GMM problem with an identity matrix as a weight to get the consistent estimate of parameters, and 
            (ii) get the weight in the way described above.
            Here, somehow authors set $q_N$ and $q_P$ and calculate the weight based on these parameters.
            If these are not derived by consistent estimates, then the weight is inconsistent and thus the estimates in the second step will be biased as well. ($q_N$ and $q_P$ seems to set by the fist-stage estimate \texttt{theta} via the same algorithm with identity matrix weighting according to the Supplementary Materials)
        \end{itemize}
      \item twoStepOptimal == 0: use an identity matrix as the weight in the ``second'' step (\texttt{eye} function)
    \end{itemize}
  \item 211-216: Obtain a new 4-dimension (length($q_N$ grids)-by-length($q_P$ grids)-by-(\# villages)-by-(\# moments))   $Dnew$ from a nested matrix $D$
  \item 223-259: Bootstrap (note: here in the bootstrap, randomly generated village weights are used, not that villages are randomly chosen with replacement)
    \begin{itemize}
      \item 226-231: Generate bootstrap weights 
        \begin{itemize}
          \item bootstrap == 0 (ie. for point estimate): same weights for all villages
          \item bootstrap == 1 (ie. for bootstrap standard errors): randomly generated weights follow exponential distribution (normalized so that the sum of weights is $1$)
        \end{itemize}
      \item 236-244: Calculate the criterion function for each ($q_N, q_P$) grid
        \begin{itemize}
          \item 240: Calculate weighted moments 
          \item 242: Calculate the criterion function
        \end{itemize}
      \item 246-254: Derive the parameter values that minimize the objective function and store the results
    \end{itemize}
\end{itemize}


\section{\texttt{divergence\_model.m}}\label{divergence_model}

``This computes the deviation of the empirical moments from the simulated ones''

\subsection*{Arguments}
\begin{itemize}
  \item X: adjacency matrix
  \item Z: covariate
  \item Betas: estimates of $\beta$'s 
  \item leaders: vector indicating leaders
  \item TakeUp: vector indicating who took up microfinance
  \item Sec: matrix indicating second neighbors (= households with distance two)
  \item theta: parameter values ($q_N$, $q_P$)
  \item m: number of moments
  \item S: number of simulations
  \item T: number of trimesters (or months/quarters)
  \item EmpRate: proportion of non-leaders who took up MF
  \item version: specification of which moments to use
\end{itemize}

\subsection*{Main part}
\begin{itemize}
  \item 14: Calculate the number of villages ($G$)
  \item 23-39: Calculate empirical and simulated moments
    \begin{itemize}
      \item 25: Calculate empirical moments, using a function \hyperref[moments]{\texttt{moments}}
      \item 28-32: Calculate simulated moments
        \begin{itemize}
          \item 30: Simulate the take-up of microfinance, using a function \hyperref[diffusion_model]{\texttt{diffusion\_model}}
          \item 31: Calculate simulated moments based on the simulated take-ups, using a function \hyperref[moments]{\texttt{moments}}
        \end{itemize}
      \item 35: Calculate the mean of simulated moments
      \item 36: Take the difference of empirical and simulated moments
    \end{itemize}
\end{itemize}

\subsection*{Outputs}
\begin{itemize}
  \item D: Calculated deviation (difference of empirical and simulated moments)
  \item TimeSim: Just \texttt{zeros(G,S)}? Not used later?
\end{itemize}

\section{\texttt{moments.m}}\label{moments}

\subsection*{Arguments}
\begin{itemize}
  \item X: adjacency matrix
  \item leaders: vector indicating leaders
  \item infected: vector indicating who took up microfinance
  \item Sec: matrix indicating second neighbors (= households with distance two)
  \item j: village index
  \item version: specification of which moments to be used
\end{itemize}

\subsection*{Main part}
\begin{itemize}
  \item 3: Declare a variable ``netstats'' which contains information of networks in each village (for the function \texttt{persistent}, see \href{https://www.mathworks.com/help/matlab/ref/persistent.html}{here})
  \item 5: size(X,1) returns the length of 1st dimension, which is 43 (villages).
  \item 7-49: If a variable ``netstats'' is already defined, then almost skip the parts, but if not yet, then calculate each statistics of the networks (7-43)
    \begin{itemize}
      \item 8: Use a function \hyperref[breadthdistRAL]{\texttt{breadthdistRAL}} to compute a reachability matrix (R) and distance matrix (D). By inputting \texttt{leaders}, the D matrix include distances from leaders. 
      \item 10-11: Calculate the minimum and average distances from leaders
      \item 13-22: Calculate the minimum distance from infected and non-infected leaders
      	\begin{itemize}
      		\item $.*$ does element-wise multiplication. 
      		\item $infected.*leaders$ is a vector with elements that take the value one for infected leaders (leaders who took up microfinance). 
      		\item $(1-infected).*leaders$ is a vector with elements that take the value one for non-infected leaders.
      	\end{itemize}
      \item 24-27: Store calculated distances (matrices) in \texttt{netstats}
      \item 29-31: Display the size of stored matrices with village index.
      \item 32-34: Display matrices with binary elements which takes the value one when minimum distances is equal to 1 with village index.
      \item 36-37: Calculate whether each household is neighbor to infected and non-infected leaders
      \item 39: Calculate the degree
      \item 40: Calculate total number of edges
      \item 41: ??
      \item 42: Calculate total number of leaders
      \item 43: Calculate the total number of leaders' edges ($leaders'*X$ is a vector of degrees of leaders, and $*leaders$ adds up those degrees)
    \end{itemize}
  \item 52-: Moments under different versions
    \begin{itemize}
      \item 53-88: Case1 (moments (2)-(6) in the paper):
        \begin{itemize}
          \item 54-62: ``Fraction of nodes that have no taking neighbors but are takers themselves'' (moment (2) in the paper)
            \begin{itemize}
              \item 56: Number of infected neighbors: \\
                \texttt{ones(N,1)*infected'}: a matrix with indicators of infected households in each row (same row vector of infected households repeated n times)\\
                $\Rightarrow$ \texttt{(ones(N,1)*infected').*X}: a matrix with indicators of infected neighbors for each household \\
                $\Rightarrow$ \texttt{sum((ones(N,1)*infected').*X, 2)}: number of infected neighbors for each household
              \item 58-59: If there is any household who are linked with other households (netstats(j).degree $>$ 0) but has no infected neighbors (infected Neighbors == 0), 
                calculate the fraction $\frac{\text{\# HH who are linked with other households, don't have any infected neighbors, but are infected themselves}}{\text{\# HH who are linked with other households and don't have any infected neighbors}}$
              \item 60-61: If there is no such household in the village, then just let the moment be $0$ (since the denominator of the above fraction will be $0$ in this case)
            \end{itemize}
          \item 64-69: ``Fraction of individuals that are infected in the neighborhood of infected leaders stats(1) = 0'' (moment (3) in the paper)
            \begin{itemize}
              \item If \texttt{sum(netstats(j).neighborOfInfected) $>$ 0} (there is at least one household who is neighbor to infected leaders), calculate the fraction \\ $\frac{\text{\# HH who are are infected in the neighborhood of infected leaders}}{\text{\# HH who are in the neighborhood of infected leaders}}$
              \item Otherwise, just let the moment be $0$ (since the denominator of the above fraction will be $0$ in this case)
            \end{itemize}
          \item 71-76: ``Fraction of individuals that are infected in the neighborhood of non-infected leaders'' (moment (4) in the paper)
            \begin{itemize}
              \item If \texttt{sum(netstats(j).neighborOfNonInfected) $>$ 0} (there is at least one household who is neighbor to non-infected leaders), calculate the fraction \\ $\frac{\text{\# HH who are are infected in the neighborhood of non-infected leaders}}{\text{\# HH who are in the neighborhood of non-infected leaders}}$
              \item Otherwise, just let the moment be $0$ (since the denominator of the above fraction will be $0$ in this case)
            \end{itemize}
          \item 78-82: ``Covariance of individuals taking with share of neighbors taking'' (moment (5) in the paper)
            \begin{itemize}
              \item 79: NonHermits: Indicator of non-isolated households 
              \item 80: ShareofTakingNeighbors: For each non-isolated household, \\
                $\frac{\text{\# infected neighbors}}{\text{\# neighbors}}$
              \item 81: NonHermitTakers: Indicator of non-isolated households who took up
              \item 82: $\text{moment} = \frac{\sum_{i: \text{non-isolated}} (\text{Take-up})_i \times (\text{Share of taking neighbors})_i}{\text{\# Non-isolated households}}$
              \item Covariance without subtracting means?
            \end{itemize}
          \item 82-88: ``Covariance of individuals taking with share of second neighbors taking'' (moment (6) in the paper)
            \begin{itemize}
              \item 86: infectedSecond: number of infected second neighbors
              \item 87: ShareofSecond: Share of infected second neighbors for non-isolated households, \\
                $\frac{\text{\# infected second neighbors}}{\text{\# first neighbors}}$ (** Why is the denominator about first neighbors? **)
              \item 88: $\text{moment} = \frac{\sum_{i: \text{non-isolated}} (\text{Take-up})_i \times (\text{Share of taking second neighbors})_i}{\text{\# Non-isolated households}}$
            \end{itemize}
        \end{itemize}
      \item 91-112: Case2 (moments (2), (5), and (6) in the paper):
          \item 92-100: ``Fraction of nodes that have no taking neighbors but are takers themselves'' (moment (2) in the paper)
          \item 102-106: ``Covariance of individuals taking with share of neighbors taking'' (moment (5) in the paper)
          \item 110-112: ``Covariance of individuals taking with share of second neighbors taking'' (moment (6) in the paper)
      \item 115-141: Case3 (moments (2), (5), and (6) in the paper, same as case 2, but purged of leader injection points):
          \item 117: a variable that denotes whether a node is a leader
      \item 143-169: Case4 (moments (2), (5), and (6) in the paper, same as case 2, but purged of all leader points):
          \item 117: a variable that denotes whether a node is a leader
          \item 166-167: Moment (6), but second neighbors who are leaders are not taken into account (see line 167)
    \end{itemize}
\end{itemize}

\subsection*{Outputs}
\begin{itemize}
	\item stats: a vector of m calculated moments for the village j
\end{itemize}

\section{\texttt{breadthdistRAL.m}}\label{breadthdistRAL}

The source of this function is the Brain Connectivity Toolbox (brain-connectivity-toolbox.net) which is a MATLAB toolbox for complex-network analysis of structural and functional brain-connectivity data sets. (See \href{https://sites.google.com/site/bctnet/}{here}). It seems like the authors modified \texttt{breadthdist.m} to add \texttt{dummies} in the arguments. \texttt{breadthdist.m} is slower but less memory intensive than "reachdist.m".

\subsection*{Arguments}
\begin{itemize}
	\item CIJ: connection/adjacency matrix
	\item dummies: Dummy vector of length N for which you wish to compute the distance and reachability matrices.
\end{itemize}

\subsection*{Outputs}
\begin{itemize}
	\item R: reachability matrix, which describes reachability between all pairs of nodes. An entry (u,v)=1 means that there exists a path from node u to node v; alternatively (u,v)=0.
	\item D: distance matrix, which contains lengths of shortest paths between all pairs of nodes. An entry (u,v) represents the length of shortest path from node u to  node v. The average shortest path length is the characteristic path length of the network.
\end{itemize}

\section{\texttt{diffusion\_model.m}}\label{diffusion_model}

\clearpage
%\bibliographystyle{apalike}
%\bibliography{audit}

\end{document}

